\chapter*{Cel pracy i streszczenie}

\section*{Cel}

Celem ninejszej pracy było stworzenie i zbadanie przydatności klasycznych modeli machine learningowych oraz modeli wykorzystujących sieci neuronowe umożliwiających analizę sentymentu --- klasyfikację czy dana recenzja ma pozytywny czy negatywny wydźwięk --- recenzji filmowych z portalu IMDB (Internet Movie DataBase).

\section*{Streszczenie}

W rozdziale pierwszym niniejszej pracy opisujemy najpierw zagadnienie sztucznej inteligencji --- rys historyczny oraz podstawowe pojęcia. Następnie omawiamy pojedynczą dziedzinę sztucznej inteligencji --- przetwarzanie języka naturalnego. Przybliżamy założenia i metody charakterystyczne dla tej dziedziny. Definiujemy też konstrukcje i metody wykorzystywane w zagadnieniach uczenia maszynowego, do których odwołują się kolejne rozdziały. W szczególności opisujemy problem klasyfikacji binarnej, który rozwiązujemy w ramach niniejszej pracy.

W drugim rozdziale pracy omawiamy wykorzystane przez nas algorytmy: las losowy, maszyna wektorów nośnych, konwolucyjne sieci neuronowe, LSTM, oraz połączenie sieci konwolucyjnej z LSTM. Dla każdego algorytmu opisana jest jego konstrukcja i właściwości, a także szczegóły implementacji wykorzystanej w obrębie pracy.

W trzecim rozdziale pracy przedstawiamy eksperymenty aplikujące poszczególne przedstawione algorytmy do wybranego zbioru danych. Najpierw omawiamy przygotowanie zbioru danych. Następnie dla każdego z algorytmów prezentujemy implementację oraz omawiamy sposób dobrania parametrów modelu. Pokazujemy też estymację przydatności poszczególnych algorytmów za pomocą wybranego zbioru metryk.

W ostatnim rozdziale znajduje się podsumowanie otrzymanych wyników.