\chapter*{Wstęp}

\subsection*{Cel}

Celem ninejszej pracy było stworzenie i zbadanie przydatności klasycznych modeli machine learningowych oraz modeli wykorzystujących sieci neuronowe umożliwiających analizę sentymentu recenzji filmowych --- klasyfikację czy dana recenzja ma pozytywny czy negatywny wydźwięk --- z portalu IMDB (Internet Movie DataBase).

\subsection*{Słowa kluczowe}

Przetwarzanie języka naturalnego, analiza sentymentu, recenzja filmowa, uczenie maszynowe, las losowy, maszyna wektorów nośnych, konwolucyjna sieć neuronowa, LSTM.

\subsection*{Streszczenie}

\indent W rozdziale pierwszym niniejszej pracy opisujemy najpierw zagadnienie sztucznej inteligencji --- rys historyczny oraz podstawowe pojęcia. Następnie omawiamy pojedynczą dziedzinę sztucznej inteligencji --- przetwarzanie języka naturalnego. Przybliżamy założenia i metody charakterystyczne dla tej dziedziny. Definiujemy też konstrukcje i metody wykorzystywane w zagadnieniach uczenia maszynowego, do których odwołują się kolejne rozdziały. W szczególności opisujemy problem klasyfikacji binarnej, który rozwiązujemy w ramach niniejszej pracy. \\

\noindent W drugim rozdziale pracy omawiamy wykorzystane przez nas algorytmy: las losowy, maszynę wektorów nośnych, konwolucyjną sieć neuronową, LSTM, oraz połączenie sieci konwolucyjnej z LSTM. Dla każdego algorytmu opisana jest jego konstrukcja i właściwości, a także szczegóły implementacji wykorzystanej w obrębie pracy. \\

\noindent W trzecim rozdziale pracy przedstawiamy eksperymenty aplikujące poszczególne przedstawione algorytmy do wybranego zbioru danych. Najpierw omawiamy przygotowanie zbioru danych. Następnie dla każdego z algorytmów prezentujemy implementację oraz omawiamy sposób dobrania parametrów modelu. Pokazujemy też estymację przydatności poszczególnych algorytmów za pomocą wybranego zbioru metryk. \\

\noindent W czwartym, ostatnim rozdziale, znajduje się podsumowanie otrzymanych wyników i refleksje końcowe.

