\documentclass[12pt]{article}
\usepackage{amssymb}
\usepackage[utf8]{inputenc}
\usepackage{polski}
\usepackage[polish]{babel}

\usepackage{hyperref} % enables using hyperlinks in captions
\usepackage{float} % allows using option [H] in the positioning of figures
\usepackage{xcolor} % allows defining own colors according to schema \definecolor{myOrange}{rgb}{1, 0.5, 0}

\usepackage{chronosys} % to make a timeline

\definecolor{TODO}{rgb}{1, 0.5, 0}
\definecolor{darkerBlue}{rgb}{0.01 ,0.2, 0.55}
\definecolor{babyblueeyes}{rgb}{0.63, 0.79, 0.95}
\usepackage{hyperref}
\begin{document}

\begin{center}
	\begin{tabular}{c c}
		\hline
		\\
		\textbf{Lidia J. Opuchlik} \qquad 	& \qquad \qquad
		\textbf{Sandra Rawicz}	\\
		
		\textbf{s16478}&\qquad \qquad \textbf{s16536 }\\
		& \\
		\hline
	\end{tabular}
\end{center}



\noindent Wkład do części pisemnej pracy inżynierskiej był następujący:

\bigskip
\noindent Cele i streszczenie --- Sandra Rawicz \\
Rozdział 1 --- Lidia J. Opuchlik \\
Rozdział 2 --- Sandra Rawicz \\
Rozdział 3 --- Lidia J. Opuchlik i Sandra Rawicz (po równo) \\
Rozdział 4 --- Lidia J. Opuchlik \\


\bigskip
\noindent Opracowanie koncepcji oraz modelowanie było wykonywane w równym stopniu przez Lidię J. Opuchlik i Sandrę Rawicz.

\bigskip\bigskip\bigskip

\begin{tabular}{p{7cm} p{5cm}}

	 & 
	 	Podpis promotora
	 \\
		& \\
		& \\
     & .............................................................. \\

\end{tabular}




\end{document}